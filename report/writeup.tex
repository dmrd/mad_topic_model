\documentclass[14pt]{article} % For LaTeX2e
\usepackage{amsmath}
\usepackage{verbatim}
\usepackage{amssymb}
\usepackage{fullpage}
\usepackage{tikz} 
\usepackage{setspace}
\usepackage{amsmath}
\usepackage{amsthm}
\usepackage{amsfonts}
\usepackage{mathrsfs}
\usepackage{subfig}
\usepackage{etex}
\reserveinserts{18}
\usepackage{morefloats}
\usepackage{dsfont}
\usepackage{tikz}
\usepackage[square, numbers]{natbib}
\usepackage[colorlinks,citecolor=red]{hyperref}
%\usepackage{algorithmicx}
%\usepackage{algorithm2e}
\usepackage{algpseudocode}
\usepackage{mathtools}
\DeclarePairedDelimiter{\ceil}{\lceil}{\rceil}

\usetikzlibrary{fit,positioning}

\theoremstyle{plain}
\newtheorem{thm}{Theorem}[section]
\newtheorem{lem}[thm]{Lemma}
\newtheorem{prop}[thm]{Proposition}
\newtheorem*{cor}{Corollary}

\theoremstyle{definition}
\newtheorem{defn}{Definition}[section]
\newtheorem{ass}{Assumption}[section]
\newtheorem{conj}{Conjecture}[section]
\newtheorem{exmp}{Example}[section]
\newtheorem{exc}{Exercise}[section]


\theoremstyle{remark}
\newtheorem*{rem}{Remark}
\newtheorem*{note}{Note}



\title{MAD Style: Multivalent Authorship Detection (MAD) Topic Models for Stylometric Analysis}

\author{David Dohan, Charles Marsh, Shubhro Saha, Max Simchowitz}
\begin{document}
\maketitle
\large
\begin{abstract}
We draw a lot on \citep{Blei2007}.
\end{abstract}

\section{Introduction}

\section{Literature Review}

\section{Test Corpi}

\section{Feature Extraction}

We incorporated six different stylometric features, each of which was composed into $n$-grams of varying sizes before being fed into the model:
\begin{enumerate}
\item Part-of-Speech (POS) tags (e.g., `Noun' for the word ``apple''). The Penn-Treebank tag set was used, and tagging itself was performed using the Maximum Entropy approach of \citet{Ratnaparkhi}.
\item Etymological tags (e.g., `Old English' for the word ``great''). Etymological information was scraped from \textit{Webster's} Dictionary \citep{Dictionary}. As etymology is inherently root-based, words absent from the dataset were first stemmatized using the method of \citet{Porter} and lemmatized using the WordNet method of \citet{Fellbaum}. If either of the results returned were present in the dictionary, their corresponding etymological tag was returned. Else, the entry with minimum Levenshtein distance was used in-place.
\item Stress.
\item Word counts.
\item Syllable counts.
\item Syllables.
\item Meter.
\end{enumerate}

\section{Methods}

\section{Evaluation}



\newpage
\bibliography{writeup}
\bibliographystyle{plainnat}

\newpage

\begin{appendix}
\end{appendix}

\end{document}



