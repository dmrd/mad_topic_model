%%%%%%%%%%%%%%%%%%%%%%%%%%%%%%%%%%%%%%%%%
% Jacobs Landscape Poster
% LaTeX Template
% Version 1.0 (29/03/13)
%
% Created by:
% Computational Physics and Biophysics Group, Jacobs University
% https://teamwork.jacobs-university.de:8443/confluence/display/CoPandBiG/LaTeX+Poster
% 
% Further modified by:
% Nathaniel Johnston (nathaniel@njohnston.ca)
%
% This template has been downloaded from:
% http://www.LaTeXTemplates.com
%
% License:
% CC BY-NC-SA 3.0 (http://creativecommons.org/licenses/by-nc-sa/3.0/)
%
%%%%%%%%%%%%%%%%%%%%%%%%%%%%%%%%%%%%%%%%%

%----------------------------------------------------------------------------------------
%	PACKAGES AND OTHER DOCUMENT CONFIGURATIONS
%----------------------------------------------------------------------------------------

\documentclass[final]{beamer}

\usepackage[scale=1.24]{beamerposter} % Use the beamerposter package for laying out the poster
\usepackage{gb4e}
\usepackage{color}
\usepackage[labelformat=empty]{caption}
\usepackage[normalem]{ulem}
\usepackage{amsfonts}
\usepackage{graphicx}
\usepackage{tikz}
\usepackage[square, numbers]{natbib}
%\usepackage{algpseudocode}
\usepackage{algorithm}

\usetikzlibrary{calc}
\usetikzlibrary{shapes}
\usetikzlibrary{arrows}
\usetikzlibrary{fit,positioning}

\usetheme{confposter} % Use the confposter theme supplied with this template

\setbeamercolor{block title}{fg=BreakfastGreen,bg=white} % Colors of the block titles
\setbeamercolor{block body}{fg=black,bg=white} % Colors of the body of blocks
\setbeamercolor{block alerted title}{fg=white,bg=BreakfastBlue!90} % Colors of the highlighted block titles

\setbeamercolor{block alerted body}{fg=black,bg=BreakfastBlue!10} % Colors of the body of highlighted blocks
% Many more colors are available for use in beamerthemeconfposter.sty

%-----------------------------------------------------------
% Define the column widths and overall poster size
% To set effective sepwid, onecolwid and twocolwid values, first choose how many columns you want and how much separation you want between columns
% In this template, the separation width chosen is 0.024 of the paper width and a 4-column layout
% onecolwid should therefore be (1-(# of columns+1)*sepwid)/# of columns e.g. (1-(4+1)*0.024)/4 = 0.22
% Set twocolwid to be (2*onecolwid)+sepwid = 0.464
% Set threecolwid to be (3*onecolwid)+2*sepwid = 0.708

\newlength{\sepwid}
\newlength{\onecolwid}
\newlength{\twocolwid}
\newlength{\threecolwid}
\setlength{\paperwidth}{48in} % A0 width: 46.8in
\setlength{\paperheight}{36in} % A0 height: 33.1in
\setlength{\sepwid}{0.024\paperwidth} % Separation width (white space) between columns
\setlength{\onecolwid}{0.22\paperwidth} % Width of one column
\setlength{\twocolwid}{0.464\paperwidth} % Width of two columns
\setlength{\threecolwid}{0.708\paperwidth} % Width of three columns
\setlength{\topmargin}{-0.5in} % Reduce the top margin size
%-----------------------------------------------------------

\usepackage{graphicx}  % Required for including images

\usepackage{booktabs} % Top and bottom rules for tables

%----------------------------------------------------------------------------------------
%	TITLE SECTION 
%----------------------------------------------------------------------------------------

\title{MAD Style: Multivalent Authorship Detection (MAD) Topic Models} % Poster title

\author{David Dohan, Charles Marsh, Shubhro Saha, Max Simchowitz} % Author(s)

\institute{Princeton University, Department of Computer Science} % Institution(s)

%----------------------------------------------------------------------------------------

\begin{document}

\addtobeamertemplate{block end}{}{\vspace*{2ex}} % White space under blocks
\addtobeamertemplate{block alerted end}{}{\vspace*{2ex}} % White space under highlighted (alert) blocks

\setlength{\belowcaptionskip}{2ex} % White space under figures
\setlength\belowdisplayshortskip{2ex} % White space under equations

\begin{frame}[t]  % The whole poster is enclosed in one beamer frame

\begin{columns}[t] % The whole poster consists of three major columns, the second of which is split into two columns twice - the [t] option aligns each column's content to the top

\begin{column}{\sepwid}\end{column} % Empty spacer column

\begin{column}{\onecolwid} % The first column

%----------------------------------------------------------------------------------------
%	OBJECTIVES
%----------------------------------------------------------------------------------------

\begin{alertblock}{Goals}
\begin{itemize}
\item Classify author writing style in a wide range of media.
\item Extract compact representation of stylistic tendency.
\item Determine which features are most indicative of writing style.
\end{itemize}
\end{alertblock}

%----------------------------------------------------------------------------------------
%	INTRODUCTION
%----------------------------------------------------------------------------------------

\begin{block}{Introduction}

In the \textit{authorship detection} problem, one is given:
\begin{itemize}
\item A set of documents labeled (by author) on which to train.
\item A set of anonymized documents to classify.
\end{itemize} Methods for authorship detection traditionally depended on careful feature extraction and rather black-box methods. Hence, they rely on extensive domain specific knowledge, and can be difficult to decipher. Here, we present the \textit{MAD Topic Model}, which uses  syntactic and stylometric n-gram features (e.g., part-of-speech tags, meter). MAD fits separate topic models to each of these ngram vocabularies, and then combines the models with a multiclass logistic regression classifier. After fitting the topic model parameters, new data can be classified using the multiclass component. 

INSERT GRAPHIC

\end{block}



%----------------------------------------------------------------------------------------

\end{column} % End of the first column

\begin{column}{\sepwid}\end{column} % Empty spacer column

\begin{column}{\twocolwid} % Begin a column which is two columns wide (column 2)

\begin{columns}[t,totalwidth=\twocolwid] % Split up the two columns wide column

\begin{column}{\onecolwid}\vspace{-.6in} % The first column within column 2 (column 2.1)

%----------------------------------------------------------------------------------------
%	MATERIALS
%----------------------------------------------------------------------------------------


\begin{block}{Model}
\begin{figure}[h!]
  \centering
  \begin{tikzpicture}
  \tikzstyle{main}=[circle, minimum size = 10mm, thick, draw =black!80, node distance = 16mm]
  \tikzstyle{connect}=[-latex, thick]
  \tikzstyle{box}=[rectangle, draw=black!100]
    \node[main, fill = white!100] (theta)  {$\theta$ };
    \node[main] (alpha) [left=of theta] { $\alpha$};
    \node[main] (z) [right=of theta] {$z$};
    \node[main, fill = black!10] (w) [right=of z] {$w$};
    \node[main] (beta) [above=of z]{$\beta$};
    \node[main] (lambda) [left=of beta]{$\lambda$};
    \node[main] (y) [right=of w, fill = white!100, xshift = -6mm]{$y$};
    \node[main] (eta) [above=of y, xshift = 28mm,yshift=-2mm]{$\eta$};
    \path (alpha) edge [connect] (theta)
          (theta) edge [connect] (z)
          (z) edge [connect] (w)
          (lambda) edge [connect] (beta)
          (beta) edge [connect] (w)
          (eta) edge [connect] (y)
          (w) edge [connect] (y);
    \node[rectangle, inner sep=8mm, fit= (z) (w),label=below right:N, yshift = 10mm, xshift=3mm] {};
	\node[rectangle, inner sep=8mm,draw=black!100, fit= (z) (w), yshift = -6mm] {};
	 \node[rectangle, inner sep=8mm, fit= (beta),label=above right:k, yshift = -14mm, xshift=-15mm] {};
	\node[rectangle, inner sep=8mm,draw=black!100, fit= (beta), yshift = 2mm] {};
	\node[rectangle, inner sep=16mm, fit= (theta) (z) (w) (y),label=below:M, yshift = 4mm, xshift=-50.5mm] {};
	\node[rectangle, inner sep=16mm, draw=black!100, fit = (theta) (z) (y) (w), yshift = -12mm, xshift=16] {};
	\node[rectangle, inner sep=20mm, fit= (alpha) (theta) (z) (w) (y),label=below left:A, yshift = 8mm, xshift=-40mm] {};
	\node[rectangle, inner sep=20mm, draw=black!100, fit = (alpha) (theta) (z) (w) (y), yshift = -13mm, xshift=16] {};
	\node[rectangle, inner sep=28mm, fit = (lambda) (beta) (alpha) (theta) (z) (w), label=above right:T, yshift = -30mm, xshift=65] {};

	\node[rectangle, inner sep=28mm, draw=black!100, fit = (lambda) (beta) (alpha) (theta) (z) (w), yshift = -13mm, xshift=16] {};

    %\node[rectangle, inner sep=4.4mm, draw=black!100, fit= (x) (iota)  (beta) (r) ,label=below right:N,yshift=-3mm ] {};
    %\node[rectangle, inner sep=3.0mm, draw=black!100, fit= (x) ,label=below left:K] {};
  \end{tikzpicture}
  \caption{Graphical Model for the MAD Topic Model}
\end{figure}
\small The MAD topic model combines the SLDA algorithm presented in \cite{wang2009simultaneous} with the Author Topic Model in \cite{rosen2004author}, and extending both to account for multiple word types. The model is variational inference, following coordinate ascent updates in \cite{wang2009simultaneous}. Stochastic variational inference was also tested, but proved impractical for these rather small data sets. 

\end{block}

%----------------------------------------------------------------------------------------

\end{column} % End of column 2.1

\begin{column}{\onecolwid}\vspace{-.6in} % The second column within column 2 (column 2.2)

%----------------------------------------------------------------------------------------
%	METHODS
%----------------------------------------------------------------------------------------

\begin{block}{Features}

\begin{figure}
\centering
\includegraphics[width=\linewidth]{dendrogram.png}
\caption{Feature extraction for the MAD Topic Model. Word and syllable counts (between punctuation) were also included.}
\end{figure}

\end{block}

%----------------------------------------------------------------------------------------

\end{column} % End of column 2.2

\end{columns} % End of the split of column 2 - any content after this will now take up 2 columns width

%----------------------------------------------------------------------------------------
%	IMPORTANT RESULT
%----------------------------------------------------------------------------------------

\begin{alertblock}{Summary}

The Multivalence Authorship Detection (MAD) Topic Model extends Latent Dirichlet Allocation \citep{Blei2003} to identify authorship in documents with many separate types (``multivalent") of count features. MAD is ``doubly supervised''--it includes a multi-class logistic regression as in \cite{Blei2007}--and also fits per-author Dirichlet distributions for each feature type. We test the MAD Topic Model on several real world corpora using a variety of $n$-gram features, including part-of-speech, syllable stress, and sequences of word lengths.
\end{alertblock} 

%----------------------------------------------------------------------------------------

\begin{columns}[t,totalwidth=\twocolwid] % Split up the two columns wide column again

\begin{column}{\onecolwid} % The first column within column 2 (column 2.1)

%----------------------------------------------------------------------------------------
%	MATHEMATICAL SECTION
%----------------------------------------------------------------------------------------

\begin{block}{Data}

To collect data for training and testing, we wrote Python scrapers for Project Gutenberg, Nassau Weekly, and Quora. 

\vspace{5 mm}

\begin{table}[ht] 
\caption{Datasets collected for training and testing}
\centering
\begin{tabular}{ c | c | c }
  Source & Authors & Docs/Author \\
  \hline
  Project Gutenberg & 5 & 50 \\
  Nassau Weekly & 550 & 200 \\
  Quora & 1600 & 100 \\
\end{tabular}
\end{table}

\vspace{10 mm}

Project Gutenberg contains excerpts from fictional books. Nassau Weekly features narrative \& editorial articles from the campus publication. Quora captures responses from top users on the question-answer site. The diversity in topic, language, and length challenges our model to detect consistent features in such a variety of contexts.

\end{block}

%----------------------------------------------------------------------------------------

\end{column} % End of column 2.1

\begin{column}{\onecolwid} % The second column within column 2 (column 2.2)

%----------------------------------------------------------------------------------------
%	RESULTS
%----------------------------------------------------------------------------------------

\begin{block}{Results}


\small
\indent Unfortunately, preliminary results show that which MAD fares far worse as using the same features with another classification scheme. This is consistent with \cite{...}, which suggests that a Pitman-Yor process better captures power law frequencies in language use than Dirichlet methods. Nevertheless, MAD's topic models over the $n$-gram stylistic features can be used to extract compact representations of stylistic tendency and discern which features are most indicative of individual writing style.

\end{block}

%----------------------------------------------------------------------------------------

\end{column} % End of column 2.2

\end{columns} % End of the split of column 2

\end{column} % End of the second column

\begin{column}{\sepwid}\end{column} % Empty spacer column

\begin{column}{\onecolwid} % The third column

%----------------------------------------------------------------------------------------
%	CONCLUSION
%----------------------------------------------------------------------------------------

\begin{block}{Visualization}

\begin{figure}
\centering
\includegraphics[width=\linewidth]{termite.png}
\caption{Topic model proportions for the word count $n$-grams. Topic 21 represents short, staccato sentences.}
\end{figure}

\end{block}

%----------------------------------------------------------------------------------------
%	ADDITIONAL INFORMATION
%----------------------------------------------------------------------------------------

\begin{block}{Conclusion}

Our (short) conclusion.

\end{block}

%----------------------------------------------------------------------------------------
%	ACKNOWLEDGEMENTS
%----------------------------------------------------------------------------------------

\setbeamercolor{block title}{fg=BreakfastRed,bg=white} % Change the block title color

\begin{block}{References}
\small
\bibliography{poster}
\bibliographystyle{plainnat}

\end{block}

\begin{center}
\includegraphics[width=0.5\linewidth]{PU-long.jpg}
\end{center}

%----------------------------------------------------------------------------------------

\end{column} % End of the third column

\end{columns} % End of all the columns in the poster

\end{frame} % End of the enclosing frame

\end{document}
