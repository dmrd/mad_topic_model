%%%%%%%%%%%%%%%%%%%%%%%%%%%%%%%%%%%%%%%%%
% Jacobs Landscape Poster
% LaTeX Template
% Version 1.0 (29/03/13)
%
% Created by:
% Computational Physics and Biophysics Group, Jacobs University
% https://teamwork.jacobs-university.de:8443/confluence/display/CoPandBiG/LaTeX+Poster
% 
% Further modified by:
% Nathaniel Johnston (nathaniel@njohnston.ca)
%
% This template has been downloaded from:
% http://www.LaTeXTemplates.com
%
% License:
% CC BY-NC-SA 3.0 (http://creativecommons.org/licenses/by-nc-sa/3.0/)
%
%%%%%%%%%%%%%%%%%%%%%%%%%%%%%%%%%%%%%%%%%

%----------------------------------------------------------------------------------------
%	PACKAGES AND OTHER DOCUMENT CONFIGURATIONS
%----------------------------------------------------------------------------------------

\documentclass[final]{beamer}

\usepackage[scale=1.24]{beamerposter} % Use the beamerposter package for laying out the poster
\usepackage{gb4e}
\usepackage{color}
\usepackage[labelformat=empty]{caption}
\usepackage[normalem]{ulem}
\usepackage{amsfonts}
\usepackage{graphicx}
\usepackage{tikz}
\usepackage[square, numbers]{natbib}
%\usepackage{algpseudocode}
\usepackage{algorithm}

\usetikzlibrary{calc}
\usetikzlibrary{shapes}
\usetikzlibrary{arrows}
\usetikzlibrary{fit,positioning}

\usetheme{confposter} % Use the confposter theme supplied with this template

\setbeamercolor{block title}{fg=BreakfastGreen,bg=white} % Colors of the block titles
\setbeamercolor{block body}{fg=black,bg=white} % Colors of the body of blocks
\setbeamercolor{block alerted title}{fg=white,bg=BreakfastBlue!90} % Colors of the highlighted block titles

\setbeamercolor{block alerted body}{fg=black,bg=BreakfastBlue!10} % Colors of the body of highlighted blocks
% Many more colors are available for use in beamerthemeconfposter.sty

%-----------------------------------------------------------
% Define the column widths and overall poster size
% To set effective sepwid, onecolwid and twocolwid values, first choose how many columns you want and how much separation you want between columns
% In this template, the separation width chosen is 0.024 of the paper width and a 4-column layout
% onecolwid should therefore be (1-(# of columns+1)*sepwid)/# of columns e.g. (1-(4+1)*0.024)/4 = 0.22
% Set twocolwid to be (2*onecolwid)+sepwid = 0.464
% Set threecolwid to be (3*onecolwid)+2*sepwid = 0.708

\newlength{\sepwid}
\newlength{\onecolwid}
\newlength{\twocolwid}
\newlength{\threecolwid}
\setlength{\paperwidth}{48in} % A0 width: 46.8in
\setlength{\paperheight}{36in} % A0 height: 33.1in
\setlength{\sepwid}{0.024\paperwidth} % Separation width (white space) between columns
\setlength{\onecolwid}{0.22\paperwidth} % Width of one column
\setlength{\twocolwid}{0.464\paperwidth} % Width of two columns
\setlength{\threecolwid}{0.708\paperwidth} % Width of three columns
\setlength{\topmargin}{-0.5in} % Reduce the top margin size
%-----------------------------------------------------------

\usepackage{graphicx}  % Required for including images

\usepackage{booktabs} % Top and bottom rules for tables

%----------------------------------------------------------------------------------------
%	TITLE SECTION 
%----------------------------------------------------------------------------------------

\title{MAD Style: Multivalent Authorship Detection (MAD) Topic Models} % Poster title

\author{David Dohan, Charles Marsh, Shubhro Saha, Max Simchowitz} % Author(s)

\institute{Princeton University, Department of Computer Science} % Institution(s)

%----------------------------------------------------------------------------------------
\makeatletter
\renewcommand{\itemize}[1][]{%
  \beamer@ifempty{#1}{}{\def\beamer@defaultospec{#1}}%
  \ifnum \@itemdepth >2\relax\@toodeep\else
    \advance\@itemdepth\@ne
    \beamer@computepref\@itemdepth% sets \beameritemnestingprefix
    \usebeamerfont{itemize/enumerate \beameritemnestingprefix body}%
    \usebeamercolor[fg]{itemize/enumerate \beameritemnestingprefix body}%
    \usebeamertemplate{itemize/enumerate \beameritemnestingprefix body begin}%
    \list
      {\usebeamertemplate{itemize \beameritemnestingprefix item}}
      {\def\makelabel##1{%
          {%
            \hss\llap{{%
                \usebeamerfont*{itemize \beameritemnestingprefix item}%
                \usebeamercolor[fg]{itemize \beameritemnestingprefix item}##1}}%
          }%
        }%
      }
  \fi%
  \beamer@cramped%
  \justifying% NEW
  %\raggedright% ORIGINAL
  \beamer@firstlineitemizeunskip%
}
\makeatother

\begin{document}

\addtobeamertemplate{block end}{}{\vspace*{2ex}} % White space under blocks
\addtobeamertemplate{block alerted end}{}{\vspace*{2ex}} % White space under highlighted (alert) blocks

\setlength{\belowcaptionskip}{2ex} % White space under figures
\setlength\belowdisplayshortskip{2ex} % White space under equations

\begin{frame}[t]  % The whole poster is enclosed in one beamer frame

\begin{columns}[t] % The whole poster consists of three major columns, the second of which is split into two columns twice - the [t] option aligns each column's content to the top

\begin{column}{\sepwid}\end{column} % Empty spacer column

\begin{column}{\onecolwid} % The first column

%----------------------------------------------------------------------------------------
%	OBJECTIVES
%----------------------------------------------------------------------------------------

\begin{alertblock}{Goals}
\begin{itemize}
\item Classify author writing style across a wide range of lexical input.
\item Determine which features are most characteristic of authors' writing styles.
\item Extract compact representations of per-author stylistic tendency.
\end{itemize}
\end{alertblock}

%----------------------------------------------------------------------------------------
%	INTRODUCTION
%----------------------------------------------------------------------------------------

\begin{block}{Introduction}

In the \textit{authorship detection} problem, one is given:
\begin{itemize}
\item A set of documents labeled (by author) on which to train.
\item A set of anonymized documents to classify.
\end{itemize} Methods for authorship detection traditionally depend on careful feature extraction and rather black-box methods. Hence, they rely on extensive domain specific knowledge, and can be difficult to decipher.

We present the \textit{MAD Topic Model}, which uses  syntactic and stylometric $n$-gram features (e.g., part-of-speech tags, meter). MAD fits separate topic models to each of these $n$-gram vocabularies and  combines the models through a multi-class logistic regression classifier. After fitting the topic model parameters, new documents can be classified using the multi-class component. As a by-product, MAD also breaks stylistic features into vocabularies over topics, creating a compact representation of stylistic tendency.

\begin{figure}
\centering
\includegraphics[width=\linewidth]{nll.png}
\caption{MAD increases Negative Log Likelihood (NLL) while training.}
\end{figure}

\end{block}



%----------------------------------------------------------------------------------------

\end{column} % End of the first column

\begin{column}{\sepwid}\end{column} % Empty spacer column

\begin{column}{\twocolwid} % Begin a column which is two columns wide (column 2)

\begin{columns}[t,totalwidth=\twocolwid] % Split up the two columns wide column

\begin{column}{\onecolwid}\vspace{-.6in} % The first column within column 2 (column 2.1)

%----------------------------------------------------------------------------------------
%	MATERIALS
%----------------------------------------------------------------------------------------


\begin{block}{Model}
\begin{figure}[h!]
  \centering
  \begin{tikzpicture}
  \tikzstyle{main}=[circle, minimum size = 10mm, thick, draw =black!80, node distance = 16mm]
  \tikzstyle{connect}=[-latex, thick]
  \tikzstyle{box}=[rectangle, draw=black!100]
    \node[main, fill = white!100] (theta)  {$\theta$ };
    \node[main] (alpha) [left=of theta] { $\alpha$};
    \node[main] (z) [right=of theta] {$z$};
    \node[main, fill = black!10] (w) [right=of z] {$w$};
    \node[main] (beta) [above=of z]{$\beta$};
    \node[main] (lambda) [left=of beta]{$\lambda$};
    \node[main, fill = white!] (y) [below=of z,  yshift = -16mm]{$y$};
    \node[main] (eta) [left=of y, xshift = -38mm]{$\eta$};
    \path (alpha) edge [connect] (theta)
          (theta) edge [connect] (z)
          (z) edge [connect] (w)
          (lambda) edge [connect] (beta)
          (beta) edge [connect] (w)
          (eta) edge [connect] (y)
          (z)   edge [connect] (y);
    %\draw[->] (z.east) to  [out=-50,in=-130] (y.west) ;
    \node[rectangle, inner sep=8mm, fit= (z) (w),label=below right:$N_d$, yshift = 10mm, xshift=-3mm] {};
	\node[rectangle, inner sep=8mm,draw=black!100, fit= (z) (w), yshift = -6mm] {};
	 \node[rectangle, inner sep=8mm, fit= (beta),label=above right:k, yshift = -14mm, xshift=-15mm] {};
	\node[rectangle, inner sep=8mm,draw=black!100, fit= (beta), yshift = 2mm] {};
	\node[rectangle, inner sep=16mm, fit= (theta) (z) (w) (y),label=right:D, yshift = -35mm, xshift=-15mm] {};
	\node[rectangle, inner sep=16mm,minimum height = 250, draw=black!100, fit = (theta) (z) (w), yshift = -27mm, xshift=16] {};
	\node[rectangle, inner sep=24mm, fit= (alpha) (theta) (z) (w) ,label=left:A, yshift = -28mm, xshift=0mm] {};
	\node[rectangle, inner sep=16mm, minimum width = 600, draw=black!100, fit = (alpha) (theta) (z) (w) , yshift = -13mm, xshift=-20] {};
	\node[rectangle, inner sep=20mm, fit = (lambda) (beta) (alpha) (theta) (z) (w), label=above right:T, yshift = -40mm, xshift=45] {};
	\node[rectangle, inner sep=16mm, minimum width = 250, draw=black!100, fit = (lambda) (beta) (alpha) (theta) (z) (w), yshift = -2mm, xshift=4] {};

    %\node[rectangle, inner sep=4.4mm, draw=black!100, fit= (x) (iota)  (beta) (r) ,label=below right:N,yshift=-3mm ] {};
    %\node[rectangle, inner sep=3.0mm, draw=black!100, fit= (x) ,label=below left:K] {};
  \end{tikzpicture}
  \caption{Graphical Model for the MAD Topic Model}
\end{figure}
The MAD Topic Model combines the SLDA algorithm presented in \cite{wang2009simultaneous} with an Author Topic Model, and extends both to account for multiple word types. The model is based on variational inference, following the coordinate ascent updates in \cite{wang2009simultaneous}. Stochastic variational inference was also tested, but proved impractical for these rather small data sets. 

\end{block}

%----------------------------------------------------------------------------------------

\end{column} % End of column 2.1

\begin{column}{\onecolwid}\vspace{-.6in} % The second column within column 2 (column 2.2)

%----------------------------------------------------------------------------------------
%	METHODS
%----------------------------------------------------------------------------------------

\begin{block}{Features}

MAD works on $n$-grams of stylistic features, such as part-of-speech tags and syllable counts.

\begin{figure}
\centering
\includegraphics[width=\linewidth]{dendrogram.png}
\caption{Feature extraction for the MAD Topic Model. Word and syllable counts (between punctuation) were also included.}
\end{figure}

\end{block}

%----------------------------------------------------------------------------------------

\end{column} % End of column 2.2

\end{columns} % End of the split of column 2 - any content after this will now take up 2 columns width

%----------------------------------------------------------------------------------------
%	IMPORTANT RESULT
%----------------------------------------------------------------------------------------
\vspace{-20mm}
\begin{alertblock}{Summary}

The Multivalent Authorship Detection (MAD) Topic Model extends Latent Dirichlet Allocation to identify authorship in documents with many separate types (``multivalent'') of count features. MAD is ``doubly supervised'': it includes a multi-class logistic regression and also fits per-author Dirichlet distributions for each feature type. We test the MAD Topic Model on several real world corpora using a variety of $n$-gram features, including part-of-speech, syllable stress, and sequences of word lengths.
\end{alertblock} 

%----------------------------------------------------------------------------------------

\begin{columns}[t,totalwidth=\twocolwid] % Split up the two columns wide column again

\begin{column}{\onecolwid} % The first column within column 2 (column 2.1)

%----------------------------------------------------------------------------------------
%	MATHEMATICAL SECTION
%----------------------------------------------------------------------------------------
\vspace{-29mm}
\begin{block}{Data}

We focused on three corpora: Project Gutenberg, Nassau Weekly, and Quora. 

\vspace{5 mm}

\begin{table}[ht] 
\caption{Datasets collected for training and testing}
\centering
\begin{tabular}{ c | c | c }
  Source & Authors & Documents \\
  \hline
  Project Gutenberg & 5 & 250 \\
  Nassau Weekly & 200 & 550 \\
  Quora & 100 & 1600 \\
\end{tabular}
\end{table}

\vspace{10 mm}

Project Gutenberg contains excerpts from fictional books. Nassau Weekly features narrative \& editorial articles from a campus publication. Quora captures responses a question-answer site. Diversity in topic and length challenges our model to detect consistent features in a variety of contexts.

\end{block}

%----------------------------------------------------------------------------------------

\end{column} % End of column 2.1

\begin{column}{\onecolwid} % The second column within column 2 (column 2.2)

%----------------------------------------------------------------------------------------
%	RESULTS
%----------------------------------------------------------------------------------------

\vspace{-29mm}
\begin{block}{Results}

Preliminary results show that MAD fares worse than alternative classification schemes. However, MAD's topic models can be used to extract compact representations of stylistic tendency.

\begin{figure}
\centering
\includegraphics[width=\linewidth]{results.png}
\end{figure}

\end{block}

%----------------------------------------------------------------------------------------

\end{column} % End of column 2.2

\end{columns} % End of the split of column 2

\end{column} % End of the second column

\begin{column}{\sepwid}\end{column} % Empty spacer column

\begin{column}{\onecolwid} % The third column

%----------------------------------------------------------------------------------------
%	CONCLUSION
%----------------------------------------------------------------------------------------

\begin{block}{Visualization}

The MAD Topic Model generates topics over $n$-grams of stylistic features, which in turn highlights the underlying stylistic structure of the documents.

\vspace{5mm}
\begin{figure}
\centering
\includegraphics[width=\linewidth]{termite.png}
\caption{A topic model for word count (between punctuation) $n$-grams. Topic 21 represents short, staccato sentences. Graphic generated with the Termite tool \citep{termite}.}
\end{figure}
\end{block}

\vspace{-5mm}

%----------------------------------------------------------------------------------------
%	ADDITIONAL INFORMATION
%----------------------------------------------------------------------------------------

\begin{block}{Conclusion}

\begin{itemize}
\item The MAD Topic Model underperformed compared to alternative classification schemes.
\item However, MAD remains useful as a means of generating topic models over stylistic features and discovering the hidden structure of natural language.
\item MAD's performance suggests that the topic model analogy may not be appropriate for stylistic features, at least in the classification setting.
\end{itemize}

\end{block}

%----------------------------------------------------------------------------------------
%	ACKNOWLEDGEMENTS
%----------------------------------------------------------------------------------------

% \setbeamercolor{block title}{fg=BreakfastRed,bg=white} % Change the block title color

\begin{block}{Acknowledgements}
Thanks to Chong Wang, whose sLDA codebase served as a starting point for the MAD Topic Model, as well as to David Blei for taking the time to consult with us on potential methods.
\end{block}

% \setbeamercolor{block title}{fg=BreakfastRed,bg=white} % Change the block title color

\begin{block}{References}
\footnotesize
\bibliography{poster}
\bibliographystyle{abbrv}

\end{block}

\vspace{-4mm}
\begin{center}
\includegraphics[width=0.5\linewidth]{PU-long.jpg}
\end{center}

%----------------------------------------------------------------------------------------

\end{column} % End of the third column

\end{columns} % End of all the columns in the poster

\end{frame} % End of the enclosing frame

\end{document}
